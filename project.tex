\documentclass[a4paper]{article}
\usepackage[utf8]{inputenc}
\usepackage[T1]{fontenc}
%\usepackage{lmodern}
\usepackage{amsmath}
\usepackage{amsfonts}
\usepackage{graphicx}
\usepackage[a4paper]{geometry}
\usepackage[frenchb]{babel}

\begin{document}
\interlinepenalty 3000
%page de garde
\setcounter{secnumdepth}{0}
\setcounter{tocdepth}{2}
\title{Infinité des nombres premiers et approximation de $\pi(x)$}
\author{Ahmed Mimouni et Hugo Spinat}
\maketitle
\tableofcontents
%fin de page de garde


\pagebreak
\begin{abstract}
  %résumé rapide en 300 mots !!!!
\end{abstract}
\pagebreak
%contenu
\part*{Introduction}
\addcontentsline{toc}{part}{Introduction}
  \subsection{Contexte}
  
  \subsection{Rappels}
    \subsubsection{Divisibilité}
    \subsubsection{Nombres premiers}
    
\part{L'infinité des nombres premiers}
  \subsection{La preuve d'Euclide}
    On considére que l'ensemble des nombres premiers est de cardinal fini $\{ p_1, ..., p_n \}$. On pose $n = p_1p_2...p_n + 1$.
    Or $n$ n'est pas un nombre premier, il posséde donc un diviseur premier $p$. Cependant, celui-ci ne peut pas appartenir aux $p_i$, car $n - p_1p_2...p_n = 1$.
    On en déduit donc qu'il existe un nombre premier $p$ qui n'appartient pas au $p_i$, et donc qu'il existe $n + 1$ nombres premiers !
    L'ensemble des nombres premiers ne peut donc pas être fini, il est donc infini.
   \subsection{Preuve par usage des nombres de Fermat}
    Examinons tout d'abord les nombres de Fermat : $F_n = 2^{2^n} + 1 $. où $n \in \mathbb{N}$. Nous allons montrer que deux nombres premiers de Fermat (distincts) sont premiers entre eux.
    Or comme ils sont premiers entre eux, cela implique que dans leur décomposition en nombre premier il existe au moins un facteur qui ne se retrouve pas dans la décomposition des autres,
    et que par conséquent il existe une infinité de nombre premier. A cette effet, vérifions la formule de récurrence : $$ \prod_{k=0}^{n-1} F_k = F_n - 2 (n \geq 1)$$
    a partir de laquelle on déduit immédiatement notre assertion. En effet, si $m$ est, par exemple, un diviseur de $F_k$ et $F_n (k < n)$ alors $m$ divise 2, et, par conséquent, $m = 1$
    ou $m = 2$. Mais $m = 2$ est impossible puisque tous les nombres de Fermat sont imparis. Pour montrer la formule, nous faisons un raisonnement par récurrence sur $n$. Pour $n = 1$,
    nous avons $F_0 = 3$ et $F_1  - 2 = 3$. Nous constatons ensuite que : $$ \prod_{k=0}^{n} F_k = \left ( \prod_{k=0}^{n - 1} F_k \right ) F_n = (F_n - 2)F_n = (2^{2^n} - 1)
    (2^{2^n} + 1) = (2^{2^{n + 1}} - 1) = F_{n + 1} - 2$$
    CQFD.
  \subsection{preuve 3}
    Soit $ \pi(x) = Card(\{p \leq x : p \in \mathbb{P}\}) $ le cardinal de l'ensemble des nombres premiers qui sont inférieurs
    ou égaux au nombre réel $x$. Enumérons les nombres premiers $\mathbb{P} = \{p_1, p_2, p_3, ...\}$ dans l'orde
    croissant. Considérons le logarithme naturel $ln x$, défini par $ln(x)= \int_1^x \frac{1}{t}dt$. Comparons
    maintenant l'aire qui se trouve sous le graphe de $f(t) = \frac{1}{t}$ avec une fonction en escalier qui
    se trouve au dessus. Si $n \leq x < n + 1$ nous avons :
    $$ln(x) \leq 1 + \frac{1}{2} + \frac{1}{3} + ... + \frac{1}{n -1} + \frac{1}{n} \leq \sum \frac{1}{m}
    \text{\indent\parbox{50ex}{où la somme s'étend à tout les $m \in \mathbb{N}^*$ 
    qui n'ont que des diviseurs premiers $p \leq x$.}}$$
    Puisque
  \subsection{preuve 4}

\part{Approximation de $\pi(x)$}

\part*{Conclusion}
\addcontentsline{toc}{part}{Conclusion}
%fin contenu
%\bibliographystyle{unsrt}
%\bibliography{biblio}

\end{document}
